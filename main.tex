% K. Caylor
%
%  Syllabus for G136 - Water, Energy, and Ecosystems, Spring Break 2019
%  Authored date: 6/10/18 
%
\documentclass[12pt]{report}
\usepackage{fullpage}
\usepackage{graphics}
% \usepackage{graphicx}
\usepackage{epsfig}

\setcounter{secnumdepth}{0}
%%%%%%%%%%%%%%%%%%%%%%%%%%%%%%%%%%%%%%%%%%%%%%%%%%%%%%%%%%%%%%%%%%%%%%%%%%%%%%%%%%%%%%%%%%%%%%%%%%%%%%%%%%%%%%%%%%%%%%%%%%%%
\begin{document}
\begin{center}
\textbf{\textsc{GEOG 136'}}

\textsc{Environmental Data Analysis}

\textsc{Department of Geography}

\textsc{Spring 2019}

\vspace{0.25in}
	\textbf{Instructor:} Kelly K. Caylor (caylor@ucsb.edu)
	
	\textbf{Associate Instructor:} Bryn Morgan (brynmorgan@ucsb.edu)
\end{center}

\section{Course Summary}

This course will provide an introduction to the principles of environmental physics and their application to ecological sciences, with a focus on programming and data analysis in Python. Course activities will use data analysis to quantify environmental patterns and processes. Emphasis will be placed developing coding skills in Python and applying these skills to environmental and biophysical problems.

\section{Course Goals} 
\begin{enumerate}
	
	\item To develop expertise in the Python programming language and the use of  Python’s data science stack to effectively store, manipulate, and gain  insight into environmental data.
	
	\item To be able to apply this understanding to characterize data on environmental patterns and processes at varying spatial and temporal scales.
	
	\item To use data to model environmental processes of energy and mass transfer.
\end{enumerate}

\section{Course Format} 
Students will learn the principles of Python programming and environmental data science by working largely independently on weekly course materials conducted in Python. Readings will be assigned for both programming and disciplinary content related to weekly themes. At least once a week, we will meet as a group to introduce and discuss concepts. In addition, students will have the opportunity to conduct weekly one-on-one check-ins with the instructional team. 

\section{Schedule of Course}

\begin{tabular}{c  | p{5.2in} }
	Weeks 1-2 & \bf{Introduction to Python Programming} \\
	          & Lab Activity: "Hello World". \\
	Weeks 3-4 & \bf{Data Science Principles in Python} \\
	           & Lab Activity: Pandas, Numpy, Matplotlib \\
    Weeks 5-6 & \bf{Statistics \& Probability} \\ 
               & Using APIs \& finding data \\
          	   & Rainfall as a Poisson Process \\
	Weeks 7-8 & \bf{Fitting Models} \\
	Weeks 9-10 & \bf{Project Work} \\
\end{tabular} 


\begin{tabular}{r p{4.25in}}
\textbf{Course Readings: } &  Readings will consist of selected book chapters, methodological surveys, and research articles. All readings will be posted in  Gauchospace.\\
\end{tabular}

\section{Evaluation}

Because of the unique circumstances of this quarter's instruction and the world in general, we intend to be very flexible with respect to assessment. In particular, we do not intend to grade each assignment beyond a rubric that is:

\begin{enumerate}
\item Was the assignment submitted?
\item Was the assignment completed?
\end{enumerate}

Students who make a good faith effort to complete all assignments will get an A in the course.

\section{Assignment Summaries}


\subsection{Weekly Exercises}

We will have regular exercises that will integrate learning goals around programming and earth system science concepts. These assignments will be designed to allow students to work independently to solve programming challenges, while also gaining more understanding and intuition into environmental data and environmental analysis. The goal of these assignments is to provide the scaffolding for project work.

\subsection{Python Library Show-and-Tell}

Each student will select a library from the python data science universe to explore in detail and present to the class. Students will develop a short tutorial that walks through some of the features of the library, and its application to earth system science. The selection of library can be broad; there are libraries for modeling, data visualization, data manipulation, data acquisition, APIs for data sources, etc... You should focus on a library that interests you. We will provide some sources for discovering libraries as the class progresses.


\subsection{Environmental Data Science Proposal and Project}

The main effort of the course will be for each student to create an independent piece of environmental analysis using principles of environmental science and python data analysis. You have wide latitude in the topics you choose, and to a large degree your choice of project will greatly influence the expertise and experience you gain from this course. Therefore, you should start by thinking about what problems interest you most, what existing areas of expertise you would like to augment or leverage. The idea of these project is to create "worked examples", i.e. demonstrations of how to integrate a question, data, and tools to gain insight into some environmental topic, problem or issue. Your project could include a focus on interactivity, data visualization, modeling, real-time analysis, etc.. but you should limit your scope to only one or two big ideas in order to make the project feasible. Each student will submit a proposal for their project by the middle of the quarter and the instructional team will work with each of you to ensure that the scope is appropriate. 


\end{document}